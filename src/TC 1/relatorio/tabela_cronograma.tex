% Use o seguinte pacote requerido no preâmbulo do seu documento:
% \usepackage{multirow} % (Inserido no preâmbulo do modelo do CCET matemática)
% \usepackage{graphicx}

\begin{table}[h]
\centering
\resizebox{\textwidth}{!}{%
\begin{tabular}{|l|l|l|l|}
\hline
\multirow{2}{*}{Atividades desenvolvidas}        & \multicolumn{3}{c|}{\begin{tabular}[c]{@{}c@{}}Bimestre\\ de\\ \imprimirdata\end{tabular}} \\ \cline{2-4} 
                                                 & \multicolumn{1}{c|}{1\textsuperscript{o}}        & \multicolumn{1}{c|}{2\textsuperscript{o}}        & 3\textsuperscript{o}                             \\ \hline
Levantamento Bibliográfico                       & \multicolumn{1}{c|}{X}        & \multicolumn{1}{c|}{X}        & \multicolumn{1}{c|}{X}        \\ \hline
Redação do Relatório do Trabalho de Curso I.     &                               & \multicolumn{1}{c|}{X}        & \multicolumn{1}{c|}{X}        \\ \hline
Apresentação do relatório do Trabalho de Curso I &                               &                               & \multicolumn{1}{c|}{X}        \\ \hline
Aplicação no ensino médio                        &                               &                               &                               \\ \hline
Estudo dos recursos computacionais               &                               &                               &                               \\ \hline
Aplicação Física                                 &                               &                               &                               \\ \hline
Redação do Relatório do Trabalho de Curso II     &                               &                               &                               \\ \hline
Revisão                                          &                               &                               &                               \\ \hline
Revisão Final                                    &                               &                               &                               \\ \hline
\end{tabular}%
}
\end{table}

% Observação:
% Essa tabela foi gerada utilizando a ferramenta online:
% 	   https://www.tablesgenerator.com
%
% O arquivo original dela está na pasta "raw/tabela_exemplo_tc.tgn".
% Importe no site "tablesgenerator" usando "File -> Load Table".