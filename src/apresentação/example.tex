\documentclass{beamer}

\usepackage[british]{babel}
\usepackage{graphicx,hyperref,ru,url}

% The title of the presentation:
%  - first a short version which is visible at the bottom of each slide;
%  - second the full title shown on the title slide;
\title[Cálculo Variacional]{Cálculo Variacional}

% Optional: a subtitle to be dispalyed on the title slide
%\subtitle{Apenas um subtítulo}

% The author(s) of the presentation:
%  - again first a short version to be displayed at the bottom;
%  - next the full list of authors, which may include contact information;
\author[Eduardo Oliveira]{Eduardo José de Oliveira}

% The institute:
%  - to start the name of the university as displayed on the top of each slide
%    this can be adjusted such that you can also create a Dutch version
%  - next the institute information as displayed on the title slide
\institute[Universidade Estadual de Goiás]{
	UNIVERSIDADE ESTADUAL DE GOIÁS\\
  	Câmpus Anápolis de Ciências Exatas e Tecnológicas Henrique Santillo \\
  	Licenciatura Em Matemática}

% Add a date and possibly the name of the event to the slides
%  - again first a short version to be shown at the bottom of each slide
%  - second the full date and event name for the title slide
\date[2019]{2019}

\begin{document}

	\begin{frame}[plain]
	  \titlepage
	\end{frame}


\begin{frame}
  %\frametitle{Outline}
  \tableofcontents
\end{frame}

% Section titles are shown in at the top of the slides with the current section 
% highlighted. Note that the number of sections determines the size of the top 
% bar, and hence the university name and logo. If you do not add any sections 
% they will not be visible.
\section{Introdução}

\begin{frame}
  \frametitle{Introdução}

  	O problema pertinente ao cálculo variacional é o de encontrar uma função diferenciável até segunda ordem $y=y(x)$ satisfazendo $y(x_1)=y_1$ e $y(x_2)=y_2$, com $x_1$, $x_2$, $y_1$ e $y_2$ dados, minimizando ou maximizando a integral
	\begin{equation}
		\int_{x_1}^{x_2} f(x,y,y')dx\text{.}
		\label{eqn:int_funcional}
	\end{equation}
\end{frame}

\section{História}

\begin{frame}
  \frametitle{História}

	\begin{itemize}
		\item Máximos e Mínimos
		\begin{itemize}
			\item Fermat (XXXX)
			\item Cálculo Diferencial
		\end{itemize}
		\item Problema da Braquistócrona
		\begin{itemize}
			\item Johann Bernoulli (XXXX) em \textit{Lorem ispum}
		\end{itemize}
		\item Cálculo Variacional
	\end{itemize}

\end{frame}

\section{Cálculo Variacional}

\begin{frame}
  \frametitle{Cálculo Variacional}

  \begin{enumerate}
    \item This just shows the effect of the style
    \item It is not a Beamer tutorial
    \item Read the Beamer manual for more help
    \item Contact me only concerning the style file
  \end{enumerate}
\end{frame}

\section{Analysis of the work}

\begin{frame}
  \frametitle{Analysis of the work}

  This style file gives your slides some nice Radboud branding.
  When you know how to work with the Beamer package it is easy to use.
  Just add:\\ ~~~$\backslash$usepackage$\{$ru$\}$ \\ at the top of your file.
\end{frame}

\section{Conclusion}

\begin{frame}
  \frametitle{Conclusion}

  \begin{itemize}
    \item Easy to use
    \item Good results
  \end{itemize}
\end{frame}

\end{document}
