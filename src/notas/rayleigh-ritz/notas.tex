\documentclass[12pt,a4paper]{article}
\usepackage[utf8]{inputenc}
\usepackage[portuguese]{babel}
\usepackage[T1]{fontenc}
\usepackage{amsmath}
\usepackage{amsfonts}
\usepackage{amssymb}
\usepackage{graphicx}
\usepackage{color}
\usepackage{tikz, tkz-euclide}
\usetkzobj{all}
\usepackage[left=3cm,right=2cm,top=3cm,bottom=2cm]{geometry}
\author{Eduardo José de Oliveira}
\title{Anotações:\\Método de Rayleigh-Ritz}

\newtheorem{example}{Exemplo}

\newcommand{\todo}[1]{{\color{red}#1}}

\begin{document}

\maketitle

No cálculo variacional, procura-se uma função $y(x)$ que dentre as funções admissíveis torna o funcional estacionário. No método de Rayleigh-Ritz, a função $y(x)$ exata é substituida por uma função aproximada $v(x)$ formada por uma combinação linear de funções $\phi_i(x)$.

Note que, a escolha adequada das funções $\phi_i(x)$ é importante para se obter uma boa aproximação para a solução do problema.

Seja o funcional 
$$
	Y=\int_{x_1}^{x_2} f(x,y,y')dx\text{,}
$$ 
com as condições de contorno $y(x_1)=0$ e $y(x_2)=0$, então,
$$
	y(x) \approx 
	v(x) =
	\sum_{i=1}^{n} a_i \phi_i (x)
	\text{,}
$$
onde $\phi_i$ são chamadas funções de forma e devem ser linearmente independente, onde cada uma delas deve satisfazer as condições de contorno
$$
	\phi_i(x_1) = 
	\phi_i(x_2) = 
	0
	\text{,}
$$
para todo $i=1$, $2$, $3$, $\dots$

Essas funções são contínuas até o grau $m-1$, sendo $m$ a ordem da maior derivada do funcional. Os coeficientes $a_i$ são chamados parâmetros de deslocamento e, finalmente, $v(x)$ é chamada de função aproximadora.

Substituindo $y$ por $v$ no funcional $Y$ e aplicando a condição de estacionáriedade, $\delta Y=0$, tem-se
$$
	\delta Y = 
	\frac{\partial Y}{\partial a_i} \delta a_i
	+
	\frac{\partial Y}{\partial a_2} \delta a_2
	+ \cdots +
	\frac{\partial Y}{\partial a_n} \delta a_n
	= 0\text{.}
$$

Como as variações $\delta a_i$ são arbitrárias, a equação se transforma em um sistema de equações homogêneas da forma
$$
	\frac{\partial Y}{\partial a_i} = 0
	\text{,}
$$
onde $i=1$, $2$, $3$, $\dots$, $n$.

\newpage

\begin{example}
	Considere a Viga prismática:
	\begin{figure}[h]
		\centering
		\begin{tikzpicture}
			\tkzInit[xmin=-0, xmax=6, ymin=-0.6, ymax=2]
			% draw axis without ticks
			\tkzDrawX[label=$x$, noticks, loosely dashed]
			\tkzDrawY[label=$v(x)$, noticks]
			
			% rectangle
			\tkzDefPoints{
				0/0.6/V_1,
				5/0.6/V_2,
				5/-0.6/V_3,
				0/-0.6/V_4}
			% complete x axis without dashed line
			\tkzDefPoints{
				5/0/A_1,
				6.35/0/A_2}
			% left base triangle
			\tkzDefPoints{
				0/-0.6/T1_1, 
				-0.25/-1.1/T1_2, 
				0.25/-1.1/T1_3}
			% right base triangle
			\tkzDefPoints{
				5/-0.6/T2_1, 
				4.75/-1.1/T2_2,
				5.25/-1.1/T2_3}
			% line on base of right triangle
			\tkzDefPoints{
				4.75/-1.2/T2_B1,
				5.25/-1.2/T2_B2}
			% length marker in bottom
			\tkzDefPoints{
				0/-1.5/L_1,
				0/-2/L_2,
				5/-1.5/L_3,
				5/-2/L_4,
				0/-1.75/LS_1,
				5/-1.75/LS_2}
			
			\tkzDrawSegment[color=black, line width=0.5mm](A_1, A_2)
			
			\tkzDrawPolygon[draw=black](T1_1, T1_2, T1_3)
			\tkzDrawPolygon[draw=black](T2_1, T2_2, T2_3)
			\tkzDrawSegment[color=black](T2_B1, T2_B2)
			\tkzDrawPolygon[draw=black](V_1, V_2, V_3, V_4)
			% q arrows
			\foreach \i in {1,...,25}{
				\tkzDefPoint(0.2*\i,1.3){A_\i}
				\tkzDefPoint(0.2*\i,0.7){B_\i}
				\draw[line width=0.6pt, black,-stealth] (A_\i) -- (B_\i);
			}
			% q text
			\node at (2.5,1.7) {$q$};
			% length marker in bottom
			\tkzDrawSegment[color=black](L_1, L_2)
			\tkzDrawSegment[color=black](L_3, L_4)
			\tkzDrawSegment[color=black](LS_1, LS_2)
			\node at (2.5,-2) {$l$};
		\end{tikzpicture}
	\end{figure}
	
	O funcional de energia potencial total associado ao problema da viga prismática é
	$$
		\Pi =
		\frac{1}{2}
		\int_0^l EI \left [ \frac{d^2 v}{dx^2} \right ]^2 dx
		-
		\int_0^l qvdx
		\text{,}
	$$
	onde $E$ é o módulo de Young do material e $I$ é o momento de inércia da seção transversal da viga em relação ao eixo baricêntrico.
	
	Como primeira aproximação, podemos utilizar a função $v_1(x)=a_0+a_1x+a_2x^2$, de onde, considerando as condições de contorno, tem-se que
	$$v_1(0)=0\Longrightarrow a_0=0\text{,}$$
	e de $v_1(l)=0$ concluí-se que $a_1=-a_2 l$. Assim, a função pode ser escrita como $$v_1(x)=a_2(x^2-lx) \text{.}$$
	
	Sua derivada é $v_1'(x)=a_2(2x-l)$ e, a derivada de segunda ordem é $v_1''(x)=2a_2$. Substituindo esses valores no funcional, tem-se que
	$$
		\Pi =
		\frac{1}{2}
		\int_0^l EI 4 a_2 ^2 dx
		-
		\int_0^l q a_2 (x^2-lx) dx
		\text{.}
	$$
	
	Da condição de estacionáriedade, $\delta \Pi=$, tem-se que
	$$
		\frac{\partial \Pi}{\partial a_0} \delta a_0
		+
		\frac{\partial \Pi}{\partial a_1} \delta a_1
		+
		\frac{\partial \Pi}{\partial a_2} \delta a_2
		= 0 \text{,}
	$$
	e, como $\Pi$ independe de $a_0$ e $a_1$, a equação se reduz a
	$$
		\frac{\partial \Pi}{\partial a_2} \delta a_2 
		= 0
		\text{,}
	$$
	de onde, devido ao fato de $\delta a_2$ ser arbitrário, obtem-se a equação
	$$
		\frac{\partial \Pi}{\partial a_2} = 0
		\text{.}
	$$
\end{example}

\end{document}