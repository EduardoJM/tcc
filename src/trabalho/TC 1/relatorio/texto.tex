\chapter{Contexto Histórico}

Para elucidar a história do Cálculo Variacional é importante mostrar um pouco da história dos máximos e mínimos de funções, problema pertinente ao cálculo diferencial, para, então, adentrar aos problemas de máximos e mínimos de funcionais, problema pertinente ao cálculo variacional.

\section{Máximos e Mínimos}

Os problemas de máximo e mínimo são corriqueiros na vida cotidiana, por exemplo, quando se quer encontrar o caminho com menor distância entre dois lugares para se caminhar uma menor distância, dentre vários outros problemas mais elaborados. Para esse exemplo específico não é necessário o uso de matemática avançada, porém, quanto mais complexidades são adicionadas aos problemas, mais a matemática é necessária para a resolução, exata ou aproximada. Para simplificar estes processos, surgem os métodos para o cálculo de máximos e mínimos das funções.

Uma das primeiras formulações matemáticas próxima das atuais para os problemas de máximos e mínimos foi feita por Pierre de Fermat (1601-1665) em 1629 considerando curvas $y=f(x)$. Ele fez comparações de $f(x)$ e $f(x+E)$ para pontos próximos. Esses valores geralmente são diferentes, porém, próximo de máximos ou mínimos a diferença se torna pequena. Deste modo, para achar os pontos de máximo ou mínimo, Fermat fazia $$\frac{f(x+E)-f(x)}{E}$$ e, após realizar a divisão, considerava $E=0$. Após considerar o valor de $E$ igual a $0$, Fermat igualava a expressão obtida a $0$, de onde conseguia extrair os valores das abscissas dos pontos de máximos e mínimos da função. \cite{boyer}

O que Fermat fez, de fato, foi igualar a primeira derivada de uma função a $0$. É importante ressaltar que esse método utilizado por Fermat veio antes mesmo da invenção do cálculo diferencial por Isaac Newton (1642-1727) em 1665-1666 e Gottfried Wilhelm Leibniz (1646-1716) em 1676, de forma independente. \cite{boyer}

\section{O Cálculo Variacional}

O ponto de partida do cálculo variacional se deu com Johann Bernoulli, em 1696, com a publicação do problema da braquistócrona no jornal científico \textit{Acta Eruditorium}. \cite{hist_courant}

O problema pode ser enunciado como:
\begin{citacao}
Sejam $A$ e $B$ dois pontos dados em um plano vertical. O problema da braquistócrona consiste em encontrar a curva que uma partícula M precisa descrever para sair de A e chegar em B no menor tempo possível, somente sob a ação da força da gravidade \cite[p. 3]{calcvar}.
\end{citacao}

O próprio Johann Bernoulli foi um dos matemáticos que solucionou o problema da braquistócrona. Ele retardou a publicação da sua solução para estimular os matemáticos do seu tempo a testarem suas habilidades nesse novo tipo de problema matemático \cite{hist_courant}. Além de Johann Bernoulli, soluções independentes foram encontradas por diversos matemáticos, como Jacob Bernoulli (1697), L'Hôpital (1697), Leibniz (1697), e Newton (1697) \cite{hist_still}.

A solução de Jacob Bernoulli considerava o aspecto da curva variável, sendo considerado o primeiro grande passo para o desenvolvimento do cálculo variacional \cite{hist_still}. Nesse problema, a quantidade a ser minimizada depende de uma curva e não apenas de uma váriavel real \cite{hist_courant}, diferentemente dos problemas relacionados ao cálculo diferencial, o que torna necessária a construção de novas ferramentas matemáticas. 

Os métodos para a resolução de problemas deste tipo eram específicos com adaptações para cada caso, sendo que os métodos gerais para a resolução só foram desenvolvidos com o envolvimento dos matemáticos Euler e Lagrange nos estudos desses problemas \cite{hist_courant}.


\chapter{Cálculo Variacional}

O problema pertinente ao cálculo variacional é o de encontrar uma função diferenciável até segunda ordem $y=y(x)$ satisfazendo $y(x_1)=y_1$ e $y(x_2)=y_2$, com $x_1$, $x_2$, $y_1$ e $y_2$ dados, minimizando ou maximizando a integral
\begin{equation}
	\int_{x_1}^{x_2} f(x,y,y')dx\text{.}
	\label{eqn:int_funcional}
\end{equation}

Para para encontrar a equação $y=y(x)$ procurada, são necessárias algumas ferramentas, dentre as quais a abordada neste estudo é a equação de Euler-Lagrange. Os conceitos, definições e resultados apresentados neste capítulo foram elaboradas segundo \citeonline{calcvar} e \citeonline{mefassan}.

\section{Equação de Euler-Lagrange}

De ínicio, é preciso demonstrar um lema que servirá como base para a dedução da equação de Euler-Lagrange.

\begin{lema}
\label{lema:cap_calcvar_lema_1}
Sejam $x_1 < x_2$ fixos e $G(x)$ uma função contínua particular para $x_1 \leqslant x \leqslant x_2$. Se $$\int_{x_1}^{x_2} \eta (x) G(x) dx = 0$$
para cada função diferenciável $\eta (x)$ tal que $\eta (x_1)=\eta (x_2)=0$, concluímos que $G(x)=0$, para todo $x$ de modo que $x_1 \leqslant x \leqslant x_2$.

\begin{proof}
Suponha que existe $\overline{x}$ tal que $x_1\ < \overline{x} < x_2$ e $G(\overline{x})\neq 0$. Podemos supor, sem perda de generalidade, que $G(\overline{x})>0$. Como $G$ é contínua, existe uma vizinhança $\overline{x_1} \leqslant \overline{x} \leqslant \overline{x_2}$ onde $G(x)>0$ em toda a vizinhança.

Podemos construir a seguinte função $\eta (x)$:
$$
\eta (x) = 
	\begin{cases}
		0 											& \mbox{para } x_1 \leqslant x < \overline{x_1}\\
		(x-\overline{x_1})^2(x-\overline{x_2})^2	& \mbox{para } \overline{x_1} \leqslant x \leqslant \overline{x_2}\\
		0											& \mbox{para } \overline{x_2} < x \leqslant x_2
	\end{cases}
$$
e então reescrever a integral do seguinte modo:
$$\int_{x_1}^{x_2}\eta (x) G(x)dx =\int_{\overline{x_1}}^{\overline{x_2}}(x-\overline{x_1})^2(x-\overline{x_2})^2G(x)dx\text{.}$$

Como $G(x) > 0$ em $\overline{x_1} \leqslant x \leqslant \overline{x_2}$, a integral do lado direito é estritamente positiva, contradizendo a hipótese. Portanto, não vale para todo $\eta (x)$, de onde $G(\overline{x})=0$. A demonstração considerando $G(\overline{x})<0$ é análogo.
\end{proof}
\end{lema}

Suponha $x_1$, $x_2$, $y_1$, $y_2$ dados, $f$ uma função de $x$, $y$ e $y'$ duas vezes diferenciável. É preciso construir uma família de funções aproximadoras, que será denotada por $Y(x)$. Essa família é definida por:
\begin{equation}\label{eqn:cap_calcvar_eq_approx}
	Y(x)=y(x)+\varepsilon \eta (x)\text{,}
\end{equation}
onde $\eta (x)$ é uma função diferenciável arbitrária para a qual $\eta (x_1)=\eta (x_2) = 0$. O número $\varepsilon$ é o parâmetro da família. É possível escrever, também, a derivada de $Y$ como
\begin{equation}\label{eqn:cap_calcvar_eq_approx_diff}
	Y'(x)=y(x)+\varepsilon \eta (x)\text{.}
\end{equation}

%\textit{\color{red} Inserir uma imagem da representação gráfica das famílias de funções aproximadoras.}

\begin{figure}[!h]
	\caption{Representação gráfica das funções aproximadoras.}
	\centering
	
	\resizebox{0.75\textwidth}{!}
	{
		\begin{tikzpicture}
	\tkzInit[xmin=-1, xmax=6, ymin=-1, ymax=6]
	% draw axis without ticks
	\tkzDrawX[noticks]
	\tkzDrawY[noticks]
	
	% define extrems on y(x) curve
	\tkzDefPoint[](2.5, 2.0){A}
	\tkzDefPoint[](4.6, 3.76){B}		
	% define helper points
	\tkzDefPoint[](2.5, 0.0){C}
	\tkzDefPoint[](4.6, 0.0){D}
	\tkzDefPoint[](0.0, 2.0){E}
	\tkzDefPoint[](0.0, 3.76){F}
	% helper points to put labels
	\tkzDefPoint[](3.4, 3.2){G}
	\tkzDefPoint[](2.6, 1.7){H}
	\tkzDefPoint[](3.7, 4.2){I}
	\tkzDefPoint[](3.2, 0.4){J}
	
	% draw extrem points on y(x) curve
	\tkzDrawPoint[fill=black, size=7](A)
	\tkzDrawPoint[fill=black, size=7](B)
		
	% y(x)
	\tkzFct[domain=2.5:4.6,samples=1000, line width=2pt]{-exp(-x+3.2)+4}
	% eta(x)
	\tkzFct[domain=2.5:4.6,samples=1000]{-0.18*x*x+1.29*x-2.09}
	% y(x) + 2*eta(x)
	\tkzFct[domain=2.5:4.6,samples=1000,dashed]{-exp(-x+3.2)+4+2*(-0.18*x*x+1.29*x-2.09)}
	% y(x) - 4*eta(x)
	\tkzFct[domain=2.5:4.6,samples=1000,dashed]{-exp(-x+3.2)+4-4*(-0.18*x*x+1.29*x-2.09)}
		
	% draw dotted segments (x_1, y_1), (x_2, y_2)
	\tkzDrawSegment[dotted](A,C)
	\tkzDrawSegment[dotted](B,D)
	\tkzDrawSegment[dotted](A,E)
	\tkzDrawSegment[dotted](B,F)
		
	% x_1, x_2, y_1, y_2 labels
	\tkzLabelPoint[below](C){$x_1$}
	\tkzLabelPoint[below](D){$x_2$}
	\tkzLabelPoint[left](E){$y_1$}
	\tkzLabelPoint[left](F){$y_2$}
	
	\tkzLabelPoint[right](G){$y(x)$}
	\tkzLabelPoint[right](H){$Y_2(x)=y(x)+\varepsilon _2 \eta(x)$}
	\tkzLabelPoint[right](I){$Y_1(x)=y(x)+\varepsilon _1 \eta(x)$}
	\tkzLabelPoint[right](J){$\eta(x)$}
\end{tikzpicture}
	}
	
	\label{fig:test_01}
	\legend{\ABNTEXfontereduzida Fonte: Elaborada pelo autor, 2019.}
\end{figure}

Reescrevendo a integral (\ref{eqn:int_funcional}) utilizando as funções aproximadoras definidas, tem-se
\begin{equation}\label{eqn:int_funcional_approx}
I(\varepsilon)=\int_{x_1}^{x_2}f(x, Y, Y')dx\text{.}
\end{equation}

Note que a função procurada $y(x)$ é o membro da família $Y(x)$ quando $\varepsilon = 0$. Ou seja, se $\varepsilon = 0$ pode-se substituir $Y$ e $Y'$ por $y$ e $y'$, respectivamente. Deste modo, a integral (\ref{eqn:int_funcional_approx}) fornece os mesmos extremos que a (\ref{eqn:int_funcional}) quando $\varepsilon=0$.

A condição necessária para que uma função de uma variável real tenha um extremo em algum ponto é que sua primeira derivada se anule nesse ponto. Então, é necessário que
\begin{equation}\label{eqn:cap_calcvar_condition}
I'(0)=0\text{.}
\end{equation}

Utilizando a Regra de Leibniz (Teorema \ref{teorema:regra_de_leibniz} do Apêndice \ref{apend:regra_de_leibniz}), a derivada de (\ref{eqn:int_funcional_approx}) pode ser escrita como
$$I'(\varepsilon)=\int_{x_1}^{x_2} \frac{\partial f}{\partial \varepsilon} (x, Y, Y') dx \text{,}$$
e, aplicando regra da cadeia, obtêm-se
$$I'(\varepsilon)=\int_{x_1}^{x_2}\left ( \frac{\partial f}{\partial x}\frac{\partial x}{\partial \varepsilon} + \frac{\partial f}{\partial Y} \frac{\partial Y}{\partial \varepsilon} + \frac{\partial f}{\partial Y'} \frac{\partial Y'}{\partial \varepsilon} \right )dx\text{,}$$
onde o primeiro termo do integrando é nulo, dado ao fato de que $x$ independe de $\varepsilon$, portanto $\dfrac{\partial x}{\partial \varepsilon}=0$, ou seja,
\begin{equation}\label{eqn:cap_calcvar_chain_rule}
I'(\varepsilon)=\int_{x_1}^{x_2}\left ( \frac{\partial f}{\partial Y}\frac{\partial Y}{\partial \varepsilon} + \frac{\partial f}{\partial Y'}\frac{\partial Y'}{\partial \varepsilon} \right ) dx \text{.}
\end{equation}

Derivando \eqref{eqn:cap_calcvar_eq_approx} em função de $\varepsilon$, tem-se $\frac{\partial Y}{\partial \varepsilon}=\frac{\partial y}{\partial \varepsilon}+\frac{\partial}{\partial \varepsilon}(\varepsilon \eta)$, de onde conclui-se que $\frac{\partial Y}{\partial \varepsilon}=\eta$, pois $y$ e $\eta$ independem de $\varepsilon$. O mesmo acontece com \eqref{eqn:cap_calcvar_eq_approx_diff}, donde verifica-se que $\frac{\partial Y'}{\partial \varepsilon}=\eta'$. Deste modo, de \eqref{eqn:cap_calcvar_chain_rule}, obtêm-se a integral
$$I'(\varepsilon)=\int_{x_1}^{x_2}\left ( 
	\frac{\partial f}{\partial Y} \eta +
	\frac{\partial f}{\partial Y'} \eta '
\right )dx \text{.}
$$

Para calcular $I'(0)$, tem-se que $\varepsilon=0$ e, então, podemos trocar $Y$ e $Y'$ por $y$ e $y'$, respectivamente, obtendo
$$
I'(0)=\int_{x_1}^{x_2}\left (
	\frac{\partial f}{\partial y} \eta +
	\frac{\partial f}{\partial y'} \eta '
\right )dx
$$
$$
I'(0)=
	\int_{x_1}^{x_2} \frac{\partial f}{\partial y}\eta dx
	+
	\int_{x_1}^{x_2} \frac{\partial f}{\partial y'}\eta' dx \text{.}
$$

Integrando o segundo membro, por partes, tomando $u=\frac{\partial f}{\partial y'}$ e $dv=\eta'dx$ de onde obtêm-se $du=\frac{d}{dx}\left ( \frac{\partial f}{\partial y'} \right )dx$ e $v=\eta$, portanto,
$$
I'(0)=
	\int_{x_1}^{x_2} \frac{\partial f}{\partial y}\eta dx
	+
	\left (
	uv \Big|_{x_1}^{x_2} - \int_{x_1}^{x_2}vdu
	\right )
$$
\begin{equation}\label{eqn:cap_calcvar_part_integration}
I'(0)=
	\int_{x_1}^{x_2} \frac{\partial f}{\partial y}\eta dx
	+
	\left (
		\frac{\partial f}{\partial y'}\eta \Biggr|_{x_1}^{x_2} - \int_{x_1}^{x_2} \frac{d}{dx}\left ( \frac{\partial f}{\partial y'} \right ) \eta dx
	\right )
\end{equation}

Sabe-se que $frac{\partial f}{\partial y'}\eta \Big |_{x_1}^{x_2}=0$, devido ao fato de que $\eta(x_1)=\eta(x_2)=0$, portanto, \eqref{eqn:cap_calcvar_part_integration} pode ser reescrita como
$$
I'(0)=
	\int_{x_1}^{x_2} \frac{\partial f}{\partial y}\eta dx
	-
	\int_{x_1}^{x_2} \frac{d}{dx} \left ( \frac{\partial f}{\partial y'} \right ) \eta dx
$$
$$
I'(0)=\int_{x_1}^{x_2}\left (
	\frac{\partial f}{\partial y} -
	\frac{d}{dx}
	\left (
		\frac{\partial f}{\partial y'}
	\right )
\right )\eta dx
$$
que, a partir da condição necessária (\ref{eqn:cap_calcvar_condition}), deve ser igualada a $0$:
$$
I'(0)=\int_{x_1}^{x_2}\left (
	\frac{\partial f}{\partial y} -
	\frac{d}{dx}
	\left (
		\frac{\partial f}{\partial y'}
	\right )
\right )\eta dx = 0	\text{.}
$$
permitindo, pelo Lema \ref{lema:cap_calcvar_lema_1}, obter a seguinte equação:
\begin{equation}\label{eqn:cap_calcvar_euler_lagrange}
\frac{\partial f}{\partial y} - \frac{d}{dx} \left ( \frac{\partial f}{\partial y'} \right )=0 \text{.}
\end{equation}

A equação diferencial parcial \eqref{eqn:cap_calcvar_euler_lagrange} é chamada de equação de Euler-Lagrange e permite encontrar uma função $f$ que maximiza ou minimiza a integral (\ref{eqn:int_funcional}).



\begin{apendicesenv}
\partapendices


% Apêndice A
\chapter{Regra de Leibniz}
\label{apend:regra_de_leibniz}
{
	Para encontrar a equação de Euler-Lagrande faz-se necessário derivar uma integral. A ferramenta matemática que permite tal feito é chamada de Regra de Leibniz (Ou Derivação sob o sinal de integral) e será enunciada e demonstrada neste Apêndice. O texto deste apêndice foi elaborado com base em \citeonline{analise_elon2}.

	\begin{definicao}
		Um conjunto $K\subset \mathbb{R}^n$ é compacto se, e somente se, toda sequência de pontos em $K$ possui uma subsequência que converge para um ponto de $K$.
	\end{definicao}

	\begin{teorema}
		\label{teorema:func_uniformemente}
		Seja $f:X\times K \longrightarrow \mathbb{R}^n$ contínua, onde $K$ é compacto. Fixemos $x_0 \in X$. Para todo $\varepsilon > 0$, existe um $\delta > 0$ tal que $x\in X$, $|x-x_0|<\delta \Longrightarrow |f(x, \alpha)-f(x_0,\alpha)|<\varepsilon$, seja qual for $\alpha \in K$.
		\begin{proof}
		Suponha que o teorema não seja válido, então existiriam $\varepsilon > 0$ e sequências de pontos $x_k\in X$, $\alpha_k \in K$ tais que $|x_k-x_0|<\frac{1}{k}$ e $|f(x_k,\alpha_k)-f(x_0,\alpha_k)|\geqslant \varepsilon$. Passando a uma subsequência, se necessário e admitindo que $\lim \alpha_k=\alpha \in K$, devido ao fato de que o conjunto $K$ é compacto.
		
		Como, $|x_k-x_0|<\frac{1}{k}$, $-\frac{1}{k}<x_k-x_0<\frac{1}{k}$, então, $x_0-\frac{1}{k}<x_k<x_0+\frac{1}{k}$, e, como $\lim \left (x_0-\frac{1}{k} \right ) = x_0$ e $\lim \left ( x_0 + \frac{1}{k} \right )=x_0$, então, $\lim x_k=x_0$. Devido a continuidade de $f$ tem-se $\varepsilon \leqslant \lim |f(x_k,\alpha_k)-f(x_0,\alpha _k)|=|f(x_0,\alpha)-f(x_0,\alpha)|=0$, uma contradição, pois da hipótese $\varepsilon >0$.
		\end{proof}
	\end{teorema}
	
	\begin{teorema}[Derivação sob o sinal de integral ou Regra de Leibniz]
		\label{teorema:regra_de_leibniz}
		Dado $U\subset \mathbb{R}^n$, aberto, seja $f:U\times[a,b]\longrightarrow \mathbb{R}$ uma função com as seguintes propriedades:
		\begin{enumerate}
			\item Para todo $x \in U$, a função $x \longmapsto f(x,t)$ é integrável em $a \leqslant t \leqslant b$.
			\item A $i$-ésima derivada parcial $\frac{\partial f}{\partial x_i}(x,t)$ existe para cada $(x,t)\in U\times [a,b]$ e a função $\frac{\partial f}{\partial x_i}:U\times [a,b]\longrightarrow \mathbb{R}$ assim definida é contínua.
		\end{enumerate}
		Então a função $\varphi: U\longrightarrow \mathbb{R}$ dada por
		$$\varphi(x)=\int_a^b f(x,t)dt\text{,}$$
		possui $i$-ésima derivada parcial em cada ponto $x\in U$, sendo
		$$\frac{\partial \varphi}{\partial x_i}(x)=\int_a^b \frac{\partial f}{\partial x_i}(x,t)dt\text{.}$$

		\begin{proof}
			Considere
			$$
			\frac{\varphi(x+se_i)-\varphi(x)}{s}
			=
			\int_a^b \frac{f(x+se_i, t)-f(x,t)}{s} dt\text{,}
			$$
			de onde subtraindo $\int_a^b \frac{\partial f}{\partial x_i}(x,t) dt$ de ambos os lados, tem-se
			\begin{equation}
			\label{eqn:anexo_leibniz_r1}
			\frac{\varphi(x+se_i)-\varphi(x)}{s}
			- \int_a^b \frac{\partial f}{\partial x_i}(x,t) dt
			=
			\int_a^b \left [ 
				\frac{f(x+se_i, t)-f(x,t)}{s} 
				- \frac{\partial f}{\partial x_i}(x,t)
			\right ] dt
			\end{equation}
			
			Pelo Teorema do Valor Médio para funções reais, existe $\theta \in [0,1]$ de modo que
			$$
				\frac{f(x+se_i,t)-f(x,t)}{s}=\frac{\partial f}{\partial x_i}(x+\theta s e_i,t)\text{,}
			$$
			onde $\theta \in [0,1]$ garante que $\theta s$ esteja entre $]0, s[$, satisfazendo as condições do Teorema do Valor Médio. Assim, de \eqref{eqn:anexo_leibniz_r1} tem-se
			%donde pode-se escrever \eqref{eqn:anexo_leibniz_r1} como
			\begin{equation}
				\label{eqn:anexo_leibniz_voltas}
				\frac{\varphi(x+se_i)-\varphi(x)}{s}
				- \int_a^b \frac{\partial f}{\partial x_i}(x,t) dt
				=
				\int_a^b \left [
					\frac{\partial f}{\partial x_i}(x+\theta s e_i, t)-\frac{\partial f}{\partial x_i}(x,t)
				\right ] dt\text{.}
			\end{equation}
			
			Como $\frac{\partial f}{\partial x_i}$ é contínua e $[a,b]$ é compacto, pelo Teorema \ref{teorema:func_uniformemente}, para todo $\varepsilon > 0$, existe um $\delta > 0$, de modo que
			\begin{equation}
				\label{eqn:anexo_leibniz_ineq_func_unif}
				|s|<\delta \Longrightarrow
				\left | 
					\frac{\partial f}{\partial x_i} (x+\theta s e_i, t) - \frac{\partial f}{\partial x_i}(x,t)
				\right | < \frac{\varepsilon}{b-a}
			\end{equation}
			seja qual for $t \in [a,b]$.
			
			Usando o fato de que $|\int_a^b f(x)dx| \leqslant \int_a^b |f(x)| dx$, obtêm-se
			$$
			\left |
				\int_a^b \left [
					\frac{\partial f}{\partial x_i}(x+\theta s e_i, t)-\frac{\partial f}{\partial x_i}(x,t)
				\right ] dt
			\right |
			\leqslant
			\int_a^b\left |
				\frac{\partial f}{\partial x_i}(x+\theta s e_i, t)-\frac{\partial f}{\partial x_i}(x,t)
			\right | dt\text{,}
			$$
			e, utilizando \eqref{eqn:anexo_leibniz_ineq_func_unif}, tem-se a inequação
			$$
			\left |
				\int_a^b \left [
					\frac{\partial f}{\partial x_i}(x+\theta s e_i, t)-\frac{\partial f}{\partial x_i}(x,t)
				\right ] dt
			\right |
			<
			\int_a^b \frac{\varepsilon}{b-a} dt
			$$
			\begin{equation}
				\label{eqn:anexo_leibniz_ineq_limit}
				\left |
					\int_a^b \left [
						\frac{\partial f}{\partial x_i}(x+\theta s e_i, t)-\frac{\partial f}{\partial x_i}(x,t)
					\right ] dt
				\right |
				< \varepsilon \text{.}
			\end{equation}
			
			De \eqref{eqn:anexo_leibniz_voltas} e \eqref{eqn:anexo_leibniz_ineq_limit}, verifica-se que
			$$
				|s|<\delta \Longrightarrow
				\left |
					\frac{\varphi(x+se_i)-\varphi(x)}{s}
					- \int_a^b \frac{\partial f}{\partial x_i}(x,t) dt
				\right |
				< \varepsilon\text{,}
			$$
			que é, a definição formal do limite
			$$
				 \lim_{s\to 0} \frac{\varphi (x+se_i)-\varphi(x)}{s}= \int_a^b \frac{\partial f}{\partial x_i} (x, t)dt \text{,}
			$$
			ou seja,
			$$
				\frac{\partial \varphi}{\partial x_i}(x)=\int _a^b \frac{\partial f}{\partial x_i}(x,t)dt\text{.}
			$$
		\end{proof}
	\end{teorema}
	
	\iffalse
	\begin{teorema}
		\label{teorema:regra_de_leibniz}
		Considere uma função $f:[a,b]\times [c,d] \longrightarrow \mathbb{R}$ e suponha que a derivada parcial $\frac{\partial f}{\partial p}(x,p)$ existe e é contínua em $[a,b]\times [c,d]$. Se a integral
		$$F(p)=\int_a^b f(x, p)dx$$
		existe para todo $p\in [c,d]$, então $F(p)$ é diferenciável e
		$$F'(p)=\int_a^b \frac{\partial f}{\partial p} (x, p)dx \text{.}$$
		\textit{\color{red} Estudar, entender e demonstrar a regra de Leibiniz.}
	\end{teorema}
	\fi
}

\end{apendicesenv}