%% (pdf)latex + bibtex + (pdf)latex + (pdf)latex [+ preview]
%%  F6 + F11 + F6 + F6 [+ F7]

%% abtex2-modelo-trabalho-academico.tex, v-1.9.6 laurocesar
%% Copyright 2012-2016 by abnTeX2 group at http://www.abntex.net.br/ 
%%
%% This work may be distributed and/or modified under the
%% conditions of the LaTeX Project Public License, either version 1.3
%% of this license or (at your option) any later version.
%% The latest version of this license is in
%%   http://www.latex-project.org/lppl.txt
%% and version 1.3 or later is part of all distributions of LaTeX
%% version 2005/12/01 or later.
%%
%% This work has the LPPL maintenance status `maintained'.
%% 
%% The Current Maintainer of this work is the abnTeX2 team, led
%% by Lauro César Araujo. Further information are available on 
%% http://www.abntex.net.br/
%%
%% This work consists of the files abntex2-modelo-trabalho-academico.tex,
%% abntex2-modelo-include-comandos and abntex2-modelo-references.bib
%%

% ------------------------------------------------------------------------
% ------------------------------------------------------------------------
% abnTeX2: Modelo de Trabalho Academico (tese de doutorado, dissertacao de
% mestrado e trabalhos monograficos em geral) em conformidade com 
% ABNT NBR 14724:2011: Informacao e documentacao - Trabalhos academicos -
% Apresentacao
% ------------------------------------------------------------------------
% ------------------------------------------------------------------------

\documentclass[
	% -- opções da classe memoir --
	12pt,				% tamanho da fonte
	openright,			% capítulos começam em pág ímpar (insere página vazia caso preciso)
    oneside,			% para impressão em recto e verso. Oposto a oneside
	a4paper,			% tamanho do papel. 
	% -- opções da classe abntex2 --
	%chapter=TITLE,		% títulos de capítulos convertidos em letras maiúsculas
	%section=TITLE,		% títulos de seções convertidos em letras maiúsculas
	%subsection=TITLE,	% títulos de subseções convertidos em letras maiúsculas
	%subsubsection=TITLE,% títulos de subsubseções convertidos em letras maiúsculas
	% -- opções do pacote babel --
	english,			% idioma adicional para hifenização
	french,				% idioma adicional para hifenização
	spanish,			% idioma adicional para hifenização
	brazil				% o último idioma é o principal do documento
	]{abntex2}


% ---
% Pacotes básicos 
% ---
\usepackage{lmodern}			% Usa a fonte Latin Modern			
\usepackage[T1]{fontenc}		% Selecao de codigos de fonte.
\usepackage[utf8]{inputenc}		% Codificacao do documento (conversão automática dos acentos)
\usepackage{lastpage}			% Usado pela Ficha catalográfica
\usepackage{indentfirst}		% Indenta o primeiro parágrafo de cada seção.
\usepackage{color}				% Controle das cores
\usepackage{graphicx}			% Inclusão de gráficos
\usepackage{microtype} 			% para melhorias de justificação
% ---
		
% ---
% Pacotes adicionais, usados apenas no âmbito do Modelo Canônico do abnteX2
% ---
\usepackage{lipsum}				% para geração de dummy text
% ---

% ---
% Pacotes para escrita matemática
% ---
\usepackage{amsmath}

\usepackage{amssymb}	 % qed
\usepackage{amsthm}      % Teoremas

%\declaretheorem[style=definition,name=Definição,parent=chapter,qed=\textemdash]{definicao}
%\declaretheorem[style=plain, name=Teorema]{teorema}
%\declaretheorem[style=plain, name=Axioma,qed=\textnormal{\textemdash}]{axioma}
%\declaretheorem[style=plain, name=Lema]{lema}
\newtheorem{teorema}{Teorema}
\newtheorem{lema}{Lema}
\newtheorem{definicao}{Definição}

% Resetar o numberador de lemas no fim de todo capítulo
% Isso faz com que a numeração seja indexada por capítulos e não
% pelo documento todo
\numberwithin{lema}{chapter}
% Idem ao de cima, com teoremas/axiomas/definições
\numberwithin{teorema}{chapter}
\numberwithin{definicao}{chapter}
%\numberwithin{axioma}{chapter}
%\numberwithin{definicao}{chapter}
% Idem aos anteriores com o ambiente figure
\numberwithin{figure}{chapter}

% PGF
\usepackage{pgf,tikz,pgfplots}
\pgfplotsset{compat=1.15}
\usepackage{mathrsfs}
\usetikzlibrary{arrows}
\usepackage{standalone}

% ---
% Pacotes de citações
% ---
\usepackage[brazilian,hyperpageref]{backref}	 % Paginas com as citações na bibl
\usepackage[alf]{abntex2cite}	% Citações padrão ABNT
% --- 
% CONFIGURAÇÕES DE PACOTES
% --- 

% ---
% Configurações do pacote backref
% Usado sem a opção hyperpageref de backref
\renewcommand{\backrefpagesname}{Citado na(s) página(s):~}
% Texto padrão antes do número das páginas
\renewcommand{\backref}{}
% Define os textos da citação
\renewcommand*{\backrefalt}[4]{
	\ifcase #1 %
		Nenhuma citação no texto.%
	\or
		Citado na página #2.%
	\else
		Citado #1 vezes nas páginas #2.%
	\fi}%
% ---

% ---
% Informações de dados para CAPA e FOLHA DE ROSTO
% ---
\titulo{Cálculo Variacional}
\autor{EDUARDO JOSÉ DE OLIVEIRA}
\local{ANÁPOLIS}
\data{2019}
\orientador{Prof. Me. Tiago de Lima Bento Pereira}
%\coorientador{Titulação e Nome do coorientador}
\instituicao{%
Universidade Estadual de Goiás
  \par
Câmpus Anápolis de Ciências Exatas e Tecnológicas Henrique Santillo
  \par
Curso de Matemática}
\tipotrabalho{Trabalho de Curso (Graduação)}
% O preambulo deve conter o tipo do trabalho, o objetivo, 
% o nome da instituição e a área de concentração 
\preambulo{Trabalho de Curso (TC) apresentado a Coordenação Adjunta de TC, como parte dos requisitos para obtenção do título de Graduado no Curso de Matemática da Universidade Estadual de Goiás.} %sob a orientação do Professor(a) titulação e nome do professor(a).}
% ---


% ---
% Configurações de aparência do PDF final

% alterando o aspecto da cor azul
\definecolor{blue}{RGB}{41,5,195}

% informações do PDF
\makeatletter
\hypersetup{
     	%pagebackref=true,
		pdftitle={\@title}, 
		pdfauthor={\@author},
    	pdfsubject={\imprimirpreambulo},
	    pdfcreator={LaTeX with abnTeX2},
		pdfkeywords={abnt}{latex}{abntex}{abntex2}{trabalho acadêmico}, 
		colorlinks=false,       		% false: boxed links; true: colored links
    	linkcolor=blue,          	% color of internal links
    	citecolor=blue,        		% color of links to bibliography
    	filecolor=magenta,      		% color of file links
		urlcolor=blue,
		bookmarksdepth=4
}
\makeatother
% --- 

% --- 
% Espaçamentos entre linhas e parágrafos 
% --- 

% O tamanho do parágrafo é dado por:
\setlength{\parindent}{1.3cm}

% Controle do espaçamento entre um parágrafo e outro:
\setlength{\parskip}{0.2cm}  % tente também \onelineskip

% ---
% compila o indice
% ---
\makeindex
% ---
\usepackage{matematicaCCETrelatorio} %MEU 
\usepackage{helvet} %Meu importa uma fonte nova "sem serifa"
\usepackage{pdfpages}

\renewcommand{\ABNTEXsectionfontsize}{\normalsize}
\usepackage{enumitem}
\usepackage{geometry}
\newgeometry{a4paper, head=95pt, left=3cm, top=4cm, right=2cm, bottom=2cm, foot=100pt}
\usepackage{multirow} % Usado para compilar a tabela da sessão "Atividades Cumpridas do Cronograma do TC"

\newcommand{\CompromissoItemOptions}{
	\begin{center}
		\begin{tabular}{c c c}
			(\hspace{0.3cm}) Sim \hspace{1.5cm}  & (\hspace{0.3cm}) Parcialmente \hspace{1.5cm} & (\hspace{0.3cm}) Não
		\end{tabular}
	\end{center}
}

\newcommand{\notitleitem}[1]{
	\item[\refstepcounter{enumi}\textbf{#1}]
}

\newcounter{apendice}[section]

\newcommand{\referencias}{\notitleitem{Referências}}
\newcommand{\apendice}[1]{
	\refstepcounter{apendice}
	\notitleitem{Apêndice~\Alph{apendice}:} #1
}

\renewcommand{\bibsection}{%
    %\chapter*{\bibname}
    \section*{REFERÊNCIAS BIBLIOGRÁFICAS}
    \bibmark
    %\ifnobibintoc\else
    %\phantomsection
    %\addcontentsline{toc}{chapter}{\texorpdfstring{\MakeTextUppercase{\bibname}}{\bibname}}
    %\fi
    \prebibhook
}

\makeoddhead{abntchapfirst}{}{}{}
\makeoddhead{abntheadings}{\ABNTEXfontereduzida\textit\rightmark}{}{}

% ----
% Início do documento
% ----
\begin{document}

% Seleciona o idioma do documento (conforme pacotes do babel)
%\selectlanguage{english}
\selectlanguage{brazil}

% Retira espaço extra obsoleto entre as frases.
\frenchspacing 

\pretextual
{
	% ---
	% Capa
	% ---
	\imprimircapa
	\clearpage
}
%
% Relatório
%
\pagestyle{ueg_mat_heading}
{
	
	%
	% Estrutura do Trabalho de Curso
	%
	\section*{ESTRUTURA DO TC}

	%Definir a estrutura inicial do TC, ou seja, quantos e quais capítulos estão previstos e o que poderá ter neste trabalho. Essa estrutura pode ser ajustada no decorrer do desenvolvimento do trabalho de curso.
	A estrutura prevista para este trabalho de curso é a seguinte:
	
	%
	% -> Ambiente Enumerate contendo a estrutura dos capítulos do TC
	%
	\begin{enumerate}[align=left,leftmargin=*,label=\textbf{Capítulo \arabic*}:]
		\item Introdução
		\item Contexto Histórico
		\item Cálculo Variacional
		\item Aplicações
		% ...
		\referencias
		\apendice {Regra de Leibniz}
	\end{enumerate}

	%
	% Atividades cumpridas do cronograma
	%
	\section*{ATIVIDADES CUMPRIDAS DO CRONOGRAMA DO TC}

	%Neste item devem ser assinaladas com um \textbf{X} as atividades que já foram desenvolvidas para a realização do TC. Lembre-se que essas atividades foram listadas no Projeto de Trabalho de Curso, no entanto, podem ser feitas alterações nessas atividades, incluindo ou excluindo algumas dessas atividades.
		
	% Use o seguinte pacote requerido no preâmbulo do seu documento:
% \usepackage{multirow} % (Inserido no preâmbulo do modelo do CCET matemática)
% \usepackage{graphicx}

\begin{table}[h!]
\centering
\resizebox{\textwidth}{!}{%
\begin{tabular}{|l|l|l|l|}
\hline
\multirow{2}{*}{Atividades desenvolvidas}        & \multicolumn{3}{c|}{\begin{tabular}[c]{@{}c@{}}Bimestre\\ de\\ \imprimirdata\end{tabular}} \\ \cline{2-4} 
                                                 & \multicolumn{1}{c|}{1\textsuperscript{o}}        & \multicolumn{1}{c|}{2\textsuperscript{o}}        & 3\textsuperscript{o}                             \\ \hline
Levantamento Bibliográfico                       & \multicolumn{1}{c|}{X}        & \multicolumn{1}{c|}{X}        & \multicolumn{1}{c|}{X}        \\ \hline
Redação do Relatório do Trabalho de Curso I.     &                               & \multicolumn{1}{c|}{X}        & \multicolumn{1}{c|}{X}        \\ \hline
Apresentação do relatório do Trabalho de Curso I &                               &                               & \multicolumn{1}{c|}{X}        \\ \hline
Aplicação no ensino médio                        &                               &                               &                               \\ \hline
Estudo dos recursos computacionais               &                               &                               &                               \\ \hline
Aplicação Física                                 &                               &                               &                               \\ \hline
Redação do Relatório do Trabalho de Curso II     &                               &                               &                               \\ \hline
Revisão                                          &                               &                               &                               \\ \hline
Revisão Final                                    &                               &                               &                               \\ \hline
\end{tabular}%
}
\end{table}

% Observação:
% Essa tabela foi gerada utilizando a ferramenta online:
% 	   https://www.tablesgenerator.com
%
% O arquivo original dela está na pasta "raw/tabela_exemplo_tc.tgn".
% Importe no site "tablesgenerator" usando "File -> Load Table". % insere o conteúdo do arquivo "tabela_cronograma.tex" apenas para simplificar a exibição\edição destes documentos
	
	%
	% Resultados Parciais
	%
	\section*{RESULTADOS PARCIAIS}
	
	Os resultados parciais são apresentados nas páginas subsequentes.
}
\clearpage
%
% Texto dos resultados parciais
%
% retoma os estilos de cabeçalho para o abntex
\thispagestyle{abntheadings}
\textual
{
	\input{texto.tex}
	
	% quebra a página para os próximos itens
	\newpage
	% retoma os estilos de cabeçalho para o modelo próprio do curso de matemática - ccet
	\thispagestyle{ueg_mat_heading}
	%
	% Referências Bibliográficas
	%
	\bibliography{abntex2-modelo-references}
	
	%Incluir somente as referências pesquisadas até o momento.

	%
	% Folha de compromisso com a execução do trabalho de curso
	%
	\imprimirfolhacompromisso
}

\end{document}