% ---
% Pacotes básicos 
% ---
\usepackage{lmodern}			% Usa a fonte Latin Modern			
\usepackage[T1]{fontenc}		% Selecao de codigos de fonte.
\usepackage[utf8]{inputenc}		% Codificacao do documento (conversão automática dos acentos)
\usepackage{lastpage}			% Usado pela Ficha catalográfica
\usepackage{indentfirst}		% Indenta o primeiro parágrafo de cada seção.
\usepackage{color}				% Controle das cores
\usepackage{graphicx}			% Inclusão de gráficos
\usepackage{microtype} 			% para melhorias de justificação
% ---
		
% ---
% Pacotes adicionais, usados apenas no âmbito do Modelo Canônico do abnteX2
% ---
\usepackage{lipsum}				% para geração de dummy text
% ---

% ---
% Pacotes para escrita matemática
% ---
\usepackage{amsmath}

\usepackage{amssymb}	 % qed
\usepackage{amsthm}      % Teoremas

\newtheorem{teorema}{Teorema}
\newtheorem{lema}{Lema}
\newtheorem{definicao}{Definição}

% Resetar o numberador de lemas no fim de todo capítulo
% Isso faz com que a numeração seja indexada por capítulos e não
% pelo documento todo
\numberwithin{lema}{chapter}
% Idem ao de cima, com teoremas/definições
\numberwithin{teorema}{chapter}
\numberwithin{definicao}{chapter}
% Idem aos anteriores com o ambiente figure
\numberwithin{figure}{chapter}

% Pacotes do Tikz utilizados para desenho
\usepackage{tikz}
\usepackage{tkz-fct}

% ---
% Pacotes de citações
% ---
\usepackage[brazilian,hyperpageref]{backref}	 % Paginas com as citações na bibl
\usepackage[alf]{abntex2cite}	% Citações padrão ABNT
% --- 
% CONFIGURAÇÕES DE PACOTES
% --- 

% ---
% Configurações do pacote backref
% Usado sem a opção hyperpageref de backref
\renewcommand{\backrefpagesname}{Citado na(s) página(s):~}
% Texto padrão antes do número das páginas
\renewcommand{\backref}{}
% Define os textos da citação
\renewcommand*{\backrefalt}[4]{
	\ifcase #1 %
		Nenhuma citação no texto.%
	\or
		Citado na página #2.%
	\else
		Citado #1 vezes nas páginas #2.%
	\fi}%
% ---

% ---
% Informações de dados para CAPA e FOLHA DE ROSTO
% ---
\titulo{\textbf{Cálculo Variacional}}
\autor{EDUARDO JOSÉ DE OLIVEIRA}
\local{Anápolis}
\data{2018}
\orientador{Prof. Me. Tiago de Lima Bento Pereira}
%\coorientador{Titulação e Nome do coorientador}
\instituicao{%
Universidade Estadual de Goiás
  \par
Câmpus Anápolis de Ciências Exatas e Tecnológicas Henrique Santillo
  \par
Curso de Matemática}
\tipotrabalho{Trabalho de Curso (Graduação)}
% O preambulo deve conter o tipo do trabalho, o objetivo, 
% o nome da instituição e a área de concentração 
\preambulo{Trabalho de Curso (TC) apresentado a Coordenação Adjunta de TC, como parte dos requisitos para obtenção do título de Graduado no Curso de Matemática da Universidade Estadual de Goiás.} %sob a orientação do Professor(a) titulação e nome do professor(a).}
% ---


% ---
% Configurações de aparência do PDF final

% alterando o aspecto da cor azul
\definecolor{blue}{RGB}{41,5,195}

% informações do PDF
\makeatletter
\hypersetup{
     	%pagebackref=true,
		pdftitle={\@title}, 
		pdfauthor={\@author},
    	pdfsubject={\imprimirpreambulo},
	    pdfcreator={LaTeX with abnTeX2},
		pdfkeywords={abnt}{latex}{abntex}{abntex2}{trabalho acadêmico}, 
		colorlinks=false,       		% false: boxed links; true: colored links
    	linkcolor=blue,          	% color of internal links
    	citecolor=blue,        		% color of links to bibliography
    	filecolor=magenta,      		% color of file links
		urlcolor=blue,
		bookmarksdepth=4
}
\makeatother
% --- 

% Define o tamanho da indentação do paragrafo
\setlength{\parindent}{1.3cm}
% Define o espaçamento entre um parágrafo e outro
\setlength{\parskip}{0.2cm}

% Adaptações para o curso de Matemática
\usepackage{abntex2matematicaCCET}
% usa a fonte helvet "sem serifa"
\usepackage{helvet}
% 
\usepackage{pdfpages}

% ---
% compila o indice
% ---
\makeindex