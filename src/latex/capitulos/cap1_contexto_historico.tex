Para elucidar a história do Cálculo Variacional é importante mostrar um pouco da história dos máximos e mínimos de funções, problema pertinente ao cálculo diferencial, para, então, adentrar aos problemas de máximos e mínimos de funcionais, problema pertinente ao cálculo variacional.

\section{Máximos e Mínimos}

Os problemas de máximo e mínimo são corriqueiros na vida cotidiana, por exemplo, quando se quer encontrar o caminho com menor distância entre dois lugares para se caminhar uma menor distância, dentre vários outros problemas mais elaborados. Para esse exemplo específico não é necessário o uso de matemática avançada, porém, quanto mais complexidades são adicionadas aos problemas, mais a matemática é necessária para a resolução, exata ou aproximada. Para simplificar estes processos, surgem os métodos para o cálculo de máximos e mínimos das funções.

Uma das primeiras formulações matemáticas próxima das atuais para os problemas de máximos e mínimos foi feita por Pierre de Fermat (1601-1665) em 1629 considerando curvas $y=f(x)$. Ele fez comparações de $f(x)$ e $f(x+E)$ para pontos próximos. Esses valores geralmente são diferentes, porém, próximo de máximos ou mínimos a diferença se torna pequena. Deste modo, para achar os pontos de máximo ou mínimo, Fermat fazia $$\frac{f(x+E)-f(x)}{E}$$ e, após realizar a divisão, considerava $E=0$. Após considerar o valor de $E$ igual a $0$, Fermat igualava a expressão obtida a $0$, de onde conseguia extrair os valores das abscissas dos pontos de máximos e mínimos da função. \cite{boyer}

O que Fermat fez, de fato, foi igualar a primeira derivada de uma função a $0$. É importante ressaltar que esse método utilizado por Fermat veio antes mesmo da invenção do cálculo diferencial por Isaac Newton (1642-1727) em 1665-1666 e Gottfried Wilhelm Leibniz (1646-1716) em 1676, de forma independente. \cite{boyer}

%\textit{\color{red}Falar de modo geral que as maiores contribuições foram de Euler e Lagrange no cálculo variacional}

\section{O Cálculo Variacional}

%Johann Bernoulli, em 1969, deu ínicio ao estudo de uma classe de problemas cujo um dos mais antigos é o problema isoperimétrico. \cite{hist_courant}. O problema isoperimétrico consiste em "[d]ado um comprimento $L>0$, encontrar, dentre todas as curvas do plano de comprimento $L$, aquela que engloba a maior área" \cite[p. 1]{cbm_isoperimetrico}.

O ponto de partida do cálculo variacional se deu com Johann Bernoulli, em 1696, com a publicação do problema da braquistócrona no jornal científico \textit{Acta Eruditorium}. \cite{hist_courant}

O problema pode ser enunciado como:
\begin{citacao}
Sejam $A$ e $B$ dois pontos dados em um plano vertical. O problema da braquistócrona consiste em encontrar a curva que uma partícula M precisa descrever para sair de A e chegar em B no menor tempo possível, somente sob a ação da força da gravidade \cite[p. 3]{calcvar}.
\end{citacao}

O próprio Johann Bernoulli foi um dos matemáticos que solucionou o problema da braquistócrona. Ele retardou a publicação da sua solução para estimular os matemáticos do seu tempo a testarem suas habilidades nesse novo tipo de problema matemático \cite{hist_courant}. Além de Johann Bernoulli, soluções independentes foram encontradas por diversos matemáticos, como Jakob Bernoulli (1697), L'Hôpital (1697), Leibniz (1697), e Newton (1697) \cite{hist_still}.

\textit{\color{red} Falar sobre a solução de Bernoulli mostrar o lado da curva variável}

O fato de que neste problema, a quantidade a ser minimizada depende de uma curva e não apenas de uma váriavel real \cite{hist_courant}, diferentemente dos problemas relacionados ao cálculo diferencial, torna necessária a construção de novas ferramentas matemáticas. 



%Os primeiros métodos para a solução destes problemas eram de caráter específico, sendo adaptados sempre a necessidade do problema em que se queria resolver até que Euler e Lagrange se envolveram no estudo destes problemas introduzindo métodos gerais para suas resoluções. \cite{hist_courant}


% Os matemáticos Leonhard Euler (1707-1783) e Joseph Louis Lagrange (1736-1813) são considerados os dois maiores matemáticos do século XVIII e suas contribuições para o cálculo variacional impulsionaram o desenvolvimento deste campo, influenciando futuras pesquisas matemáticas. \cite{hist_eves}
